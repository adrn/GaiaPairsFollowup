\documentclass[modern, letterpaper]{aastex61}

% TODO:
% -

% style notes
% -----------
% - Use the Makefile; don't be typing ``pdflatex'' or some bullshit.
% - Line break between sentences to make the git diffs readable.
% - Use \, as a multiply operator.
% - Reserve () for function arguments; use [] or {} for outer shit.
% - Use \sectionname not Section, \figname not Figure, \documentname not Article

\include{gitstuff}
% Load common packages

\usepackage{amsmath}
\usepackage{amsfonts}
\usepackage{amssymb}
\usepackage{booktabs}

\usepackage{graphicx}
\usepackage{color}

\usepackage{hyperref}
\definecolor{linkcolor}{rgb}{0.02,0.35,0.55}
\definecolor{citecolor}{rgb}{0.4,0.4,0.4}
\hypersetup{colorlinks=true,linkcolor=linkcolor,citecolor=citecolor,
            filecolor=linkcolor,urlcolor=linkcolor}
\hypersetup{pageanchor=false}

\newcommand{\documentname}{\textsl{Article}}
\newcommand{\sectionname}{Section}
\newcommand{\figname}{Figure}
\newcommand{\eqname}{Equation}
\newcommand{\tblname}{Table}

% Packages / projects / programming
\newcommand{\package}[1]{\texttt{#1}}
\newcommand{\acronym}[1]{{\small{#1}}}
\newcommand{\github}{\package{GitHub}}
\newcommand{\python}{\package{Python}}

% Missions
\newcommand{\project}[1]{\textsl{#1}}

% For referee
\newcommand{\changes}[1]{{\color{red} #1}}

% Stats / probability
\newcommand{\given}{\,|\,}
\newcommand{\norm}{\mathcal{N}}

% Maths
\newcommand{\dd}{\mathrm{d}}
\newcommand{\transpose}[1]{{#1}^{\!\mathsf{T}}}
\newcommand{\inverse}[1]{{#1}^{-1}}
\newcommand{\argmin}{\operatornamewithlimits{argmin}}
\newcommand{\argmax}{\operatornamewithlimits{argmax}}
\newcommand{\mean}[1]{\left< #1 \right>}
\renewcommand{\vec}[1]{\bs{#1}}
\newcommand{\mat}[1]{\mathbf{#1}}

% Unit shortcuts
\newcommand{\msun}{\ensuremath{\mathrm{M}_\odot}}
\newcommand{\kms}{\ensuremath{\mathrm{km}~\mathrm{s}^{-1}}}
\newcommand{\au}{\ensuremath{\mathrm{au}}}
\newcommand{\pc}{\ensuremath{\mathrm{pc}}}
\newcommand{\kpc}{\ensuremath{\mathrm{kpc}}}
\newcommand{\kmskpc}{\ensuremath{\mathrm{km}~\mathrm{s}^{-1}~\mathrm{kpc}^{-1}}}

% Misc.
\newcommand{\bs}[1]{\boldsymbol{#1}}

% Astronomy
\newcommand{\DM}{{\rm DM}}
\newcommand{\feh}{\ensuremath{{[{\rm Fe}/{\rm H}]}}}
\newcommand{\df}{\acronym{DF}}

% TO DO
\newcommand{\todo}[1]{{\color{red} TODO: #1}}


% Project-specific
\newcommand{\gaia}{\project{Gaia}}
\newcommand{\DR}[1]{\acronym{DR}1}
\newcommand{\tgas}{\acronym{TGAS}}

\begin{document}

\title{Co-moving stars in \textsl{Gaia DR1}:
       a catalog of genuine co-moving star pairs from low-resolution
       spectroscopic follow-up}

\author{Adrian~M.~Price-Whelan}
\affiliation{Department of Astrophysical Sciences,
             Princeton University, Princeton, NJ 08544, USA}
\affiliation{To whom correspondence should be addressed:
             \texttt{adrn@astro.princeton.edu}}

\author{Semyeong~Oh}
\affiliation{Department of Astrophysical Sciences,
             Princeton University, Princeton, NJ 08544, USA}

\author{Melissa~Ness}
\affiliation{todo}

\author{David~N.~Spergel}
\affiliation{Flatiron Institute,
             Simons Foundation,
             162 Fifth Avenue,
             New York, NY 10010, USA}
\affiliation{Department of Astrophysical Sciences,
             Princeton University, Princeton, NJ 08544, USA}

% \author{David~W.~Hogg}
% \affiliation{todo}

\begin{abstract}
TODO:

\end{abstract}

\keywords{
    TODO
}

\section{Introduction}\label{sec:introduction}

Outline:
- Stars form in clusters, chemically homogeneous at some level
- Clusters generally disrupt slowly through a process that preserves initial
  clustering in conserved quantities but destroys phase coherence
- Most significant process is probably tidal heating
- Dissolved clusters are therefore useful dynamical tracers of mass distribution
- In Galactic halo, globular clusters
- In disk, hope is to use chemical signatures to connect stars, then can use
  dynamics to reconstruct birth clusters, infer mass distribution / evolution
- Clusters have advantage of many stars per cluster, but fewer clusters
- Wide binary stars are little potentiometers themselves - only N=2, but are far
  more numerous as a population
- Long-term goal is to study the separation distribution of wide-binaries as a
  function of age and Galactic orbit distribution (disk vs. halo)
    - Is there a separation truncation?
    - Can that be described just from Galactic tides, or need perturbers?
- Also useful for stellar models: should have same age, chemistry, but need to
  test this
    - Chemical homogeneity tested with few open clusters - wide binaries another
      test
    - Need sample of confirmed widely-separated binaries, follow-up to get
      detailed abundances
    - Need way to verify that widely separated stars are physically associated,
      via kinematics
- Some bullshit about Gaia, TGAS, etc., released as a part of \gaia\ \DR{1}\

Most stars form in clusters with small dispersions in position, velocity, and
(likely) small initial spreads in chemical abundances.
Once a given cluster is visibly dispersing or fully dispersed, the former member
stars retain chemical and kinematic signatures of their birth cluster

, though
the kinematic mapping is non-trivial because of phase-mixing.

become
useful

Widely-separated co-moving stars are expected to form from the dissolution of
birth star clusters


\section{Data}\label{sec:data}

\subsection{Sample selection}\label{sec:sample}

The wide-separation co-moving stars presented later in this work were originally
identified as candidate co-moving stars in a search of the \tgas\ catalog.

Target sample taken from ... Oh et al.

Briefly, ...

TGAS roughly XX\% complete to 200 pc with parallax SNR cut.

Oh catalog have a high likelihood-ratio cut, but that should be unbiased (right?)

Randomly sample from that, subject to observational constraints.

\subsection{Observations and Data reduction}\label{sec:reduction}

See: http://iopscience.iop.org/article/10.1088/0004-6256/146/6/161/pdf for an example

TODO:
- Describe pipeline...

- Comparing repeat visits on subsequent nights, systematic ~20 km/s
    - see notebook "Compare RVs with previous measurements"

\section{Results}

\subsection{Catalog ...}\label{sec:}

\subsection{Physical properties of genuine co-moving pairs}\label{sec:}

\subsection{Galactic orbits of the co-moving pairs}\label{sec:}

- Orbits of true pairs in the Galaxy, any halo or thick-disk?

\section{Conclusions }



\acknowledgements
David W. Hogg (NYU)

\bibliographystyle{aasjournal}
\bibliography{refs}

\end{document}
