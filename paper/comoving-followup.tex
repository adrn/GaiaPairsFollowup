\documentclass[modern, letterpaper]{aastex61}

% TODO:
% -

% style notes
% -----------
% - Use the Makefile; don't be typing ``pdflatex'' or some bullshit.
% - Line break between sentences to make the git diffs readable.
% - Use \, as a multiply operator.
% - Reserve () for function arguments; use [] or {} for outer shit.
% - Use \sectionname not Section, \figname not Figure, \documentname not Article

\include{gitstuff}
% Load common packages

\usepackage{amsmath}
\usepackage{amsfonts}
\usepackage{amssymb}
\usepackage{booktabs}

\usepackage{graphicx}
\usepackage{color}

\usepackage{hyperref}
\definecolor{linkcolor}{rgb}{0,0,0.2}
\hypersetup{colorlinks=true,linkcolor=linkcolor,citecolor=linkcolor,
            filecolor=linkcolor,urlcolor=linkcolor}
\hypersetup{pageanchor=false}

\newcommand{\documentname}{\textsl{Article}}
\newcommand{\sectionname}{Section}
\newcommand{\figname}{Figure}
\newcommand{\eqname}{Equation}
\newcommand{\tblname}{Table}

% Packages / projects / programming
\newcommand{\package}[1]{\textsl{#1}}
\newcommand{\acronym}[1]{{\small{#1}}}
\newcommand{\github}{\package{GitHub}}
\newcommand{\python}{\package{Python}}

% Missions
\newcommand{\project}[1]{\textsl{#1}}

% For referee
\newcommand{\changes}[1]{{\color{red} #1}}

% Stats / probability
\newcommand{\given}{\,|\,}
\newcommand{\norm}{\mathcal{N}}

% Maths
\newcommand{\dd}{\mathrm{d}}
\newcommand{\transpose}[1]{{#1}^{\mathsf{T}}}
\newcommand{\inverse}[1]{{#1}^{-1}}
\newcommand{\argmin}{\operatornamewithlimits{argmin}}
\newcommand{\mean}[1]{\left< #1 \right>}

% Unit shortcuts
\newcommand{\msun}{\ensuremath{\mathrm{M}_\odot}}
\newcommand{\kms}{\ensuremath{\mathrm{km}~\mathrm{s}^{-1}}}
\newcommand{\au}{\ensuremath{\mathrm{au}}}
\newcommand{\pc}{\ensuremath{\mathrm{pc}}}
\newcommand{\kpc}{\ensuremath{\mathrm{kpc}}}
\newcommand{\kmskpc}{\ensuremath{\mathrm{km}~\mathrm{s}^{-1}~\mathrm{kpc}^{-1}}}

% Misc.
\newcommand{\bs}[1]{\boldsymbol{#1}}

% Astronomy
\newcommand{\DM}{{\rm DM}}
\newcommand{\feh}{\ensuremath{{[{\rm Fe}/{\rm H}]}}}
\newcommand{\df}{\acronym{DF}}

% TO DO
\newcommand{\todo}[1]{{\color{red} TODO: #1}}


% Project-specific
\newcommand{\gaia}{\project{Gaia}}
\newcommand{\DR}[1]{\acronym{DR}1}
\newcommand{\tgas}{\acronym{TGAS}}

\begin{document}

\title{Co-moving stars in \textsl{Gaia DR1}:
       a catalog of genuine comoving star pairs from low-resolution
       spectroscopic follow-up}

\author{Adrian~M.~Price-Whelan}
\affiliation{Department of Astrophysical Sciences,
             Princeton University, Princeton, NJ 08544, USA}
\affiliation{To whom correspondence should be addressed:
             \texttt{adrn@astro.princeton.edu}}

\author{Semyeong~Oh}
\affiliation{Department of Astrophysical Sciences,
             Princeton University, Princeton, NJ 08544, USA}

\author{Melissa~Ness}
\affiliation{todo}

\author{David~N.~Spergel}
\affiliation{Flatiron Institute,
             Simons Foundation,
             162 Fifth Avenue,
             New York, NY 10010, USA}
\affiliation{Department of Astrophysical Sciences,
             Princeton University, Princeton, NJ 08544, USA}

% \author{David~W.~Hogg}
% \affiliation{todo}

\begin{abstract}
% Context
Widely-separated, co-eval stars are [...] for studying the chemical homogeneity
of birth, calibrating stellar atmosphere models across a range of [..], and for
modeling the Galactic tidal field.
Previous work using Hipparcos + literature RVs -> a few hundred genuine pairs
out to 8 pc.
% Aims
Identified candidate pairs using the Tycho-Gaia Astrometric Solution catalog,
Gaia DR1.
Want to identify genuinely dynamically associated pairs for future
high-resolution followup
% Methods
Low-resolution spectrograph to measure relative radial velocities to identify
genuine pairs.
% Results
Of the XX pairs observed, we find YY ...
% Conclusions
Gaia DR2 -> how many do we expect?
\end{abstract}

\keywords{
    TODO
}

\section{Introduction}\label{sec:introduction}

Outline:
- Stars form in clusters, chemically homogeneous at some level
- Clusters generally disrupt slowly through a process that preserves initial
  clustering in conserved quantities but destroys phase coherence
- Most significant process is probably tidal heating
- Dissolved clusters are therefore useful dynamical tracers of mass distribution
- In Galactic halo, globular clusters
- In disk, hope is to use chemical signatures to connect stars, then can use
  dynamics to reconstruct birth clusters, infer mass distribution / evolution
- Clusters have advantage of many stars per cluster, but fewer clusters
- Wide binary stars are little potentiometers themselves - only N=2, but are far
  more numerous as a population
- Long-term goal is to study the separation distribution of wide-binaries as a
  function of age and Galactic orbit distribution (disk vs. halo)
    - Is there a separation truncation?
    - Can that be described just from Galactic tides, or need perturbers?
- Also useful for stellar models: should have same age, chemistry, but need to
  test this
    - Chemical homogeneity tested with few open clusters - wide binaries another
      test
    - Need sample of confirmed widely-separated binaries, follow-up to get
      detailed abundances
    - Need way to verify that widely separated stars are physically associated,
      via kinematics
- Some bullshit about Gaia, TGAS, etc., released as a part of \gaia\ \DR{1}\

Most stars form in clusters with small dispersions in position, velocity, and
(likely) small initial spreads in chemical abundances.
Once a given cluster is visibly dispersing or fully dispersed, the former member
stars retain chemical and kinematic signatures of their birth cluster

comoving not wide binary (don't know if they are bound)

, though
the kinematic mapping is non-trivial because of phase-mixing.

become
useful

Widely-separated comoving stars are expected to form from the dissolution of
birth star clusters


\section{Data}\label{sec:data}

\subsection{Sample selection}\label{sec:sample}

In previous work, we have identified a sample of high-confidence, candidate
comoving star pairs (\citealt{Oh:2017}) using only astrometric information from
the \tgas\ catalog (\citealt{tgas}).
These comoving pairs were selected using a cut on a marginalized likelihood
ratio computed for all nearly co-spatial pairs of stars.
For a full description of the selection method, see \citealt{Oh:2017}.
Briefly, likelihood 1, $L_1$, considers a model in which the full-space
velocities of each star in the pair are identical and sampled from a disk-like,
Sun-relative velocity distribution, and likelihood 2, $L_2$, considers a model
in which the full-space velocities of each star were drawn independently from
the same velocity distribution used above; the likelihood ratio was taken as
$L_1/L_2$.
After a parallax signal-to-noise cut ($\pi/\sigma_\pi > 8$), we compute the
likelihood ratio for all stars within $10~\pc$ of each target star from the
\tgas\ catalog.
The likelihoods appropriately take into account the reported covariances and
uncertainties between the astrometric measurements in the \tgas\ catalog and
naturally handle geometric or projection effects in comparing spherical velocity
components for star pairs separated by large angles on the sky.
After applying a conservative cut on the likelihood ratio motivated by computing
the same likelihood ratio for random pairings of stars, we find 13,058
comoving star pairs with physical separations between $1000~\au$ and $10~\pc$.
Of these pairs, there are 10,606 unique sources, implying that a significant
fraction of the identified pairs are members of larger groups.

TGAS $\approx$90\% complete to 200 pc with parallax SNR cut (\citealt{Bovy:2017}).

Oh catalog have a high likelihood-ratio cut, but that should be unbiased (right?)

Randomly sample from that, subject to observational constraints.

\subsection{Observations and Data reduction}\label{sec:reduction}

See: http://iopscience.iop.org/article/10.1088/0004-6256/146/6/161/pdf for an example

TODO:
- Describe pipeline...

- Comparing repeat visits on subsequent nights, systematic ~20 km/s
    - see notebook "Compare RVs with previous measurements"

\section{Results}

TODO:
- Plot of magnitude of 3d velocity difference vs. 3d separation (inferred).

\subsection{Catalog ...}\label{sec:}

\subsection{Physical properties of genuine comoving pairs}\label{sec:}

\subsection{Galactic orbits of the co-moving pairs}\label{sec:}

- Orbits of true pairs in the Galaxy, any halo or thick-disk?

\section{Conclusions }



\acknowledgements
David W. Hogg (NYU)

\bibliographystyle{aasjournal}
\bibliography{refs}

\end{document}
